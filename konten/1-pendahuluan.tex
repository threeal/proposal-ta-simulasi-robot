\section{PENDAHULUAN}

\subsection{Latar Belakang}

Selama beberapa tahun terakhir, robot telah mengalami perkembangan yang signifikan dari robot beroda untuk edukasi \citep{Goncalves2009} hingga robot manipulator untuk skala industri \citep{Blatnicky2020}.
Salah satu bentuk perkembangan lain dari robot tersebut adalah \emph{socially assistive robots} (SARs).
SARs merupakan jenis robot dalam bidang \emph{socially assistive robotics} yang menggabungkan aspek yang ada pada \emph{assistive robotics} dan \emph{socially interactive robotics} sehingga menjadikan SARs sebagai robot yang mampu memberikan bantuan kepada pengguna dalam bentuk interaksi sosial \citep{Seifer2005}.

Namun, karena sifat dari SARs yang melibatkan interaksi langsung dengan pengguna, maka, pengujian dari robot akan menjadi sulit dan beresiko bagi pengguna yang ikut terlibat dalam pengujian tersebut \citep{Erickson2020}.
Salah satu solusi untuk mengatasi hal tersebut adalah dengan melakukan pengujian secara virtual melalui simulasi robot.
Selain bisa meminimalisir resiko, penggunaan simulasi robot sebagai media pengujian robot juga bisa mengurangi biaya yang dibutuhkan dan menghemat waktu pengujian selama pengembangan robot tersebut \citep{Takaya2016}.

Hingga saat ini sudah ada beberapa simulator yang bisa digunakan untuk menjalankan simulasi robot seperti Webots \citep{Michel2004}, Gazebo \citep{Koenig2004}, V-REP \citep{Rohmer2013}, OpenAI Gym \citep{Brockman2016}, dan lain sebagainya.
Namun, simulator-simulator yang sudah dijelaskan sebelumnya hanyalah platform yang secara umum digunakan untuk membantu pengembangan robot melalui simulasi virtual.
Sedangkan pengembangan dari lingkungan simulasi dan controller robot untuk simulasi tersebut harus dibuat sendiri oleh pengembang robot.

Untuk itu, pada tugas akhir ini saya mengajukan penelitian terkait pengembangan lingkungan simulasi untuk pengujian SARs menggunakan ROS 2 dan Gazebo.
ROS 2 dan Gazebo sendiri dipilih karena tersedianya banyak library yang dapat membantu pengembangan maupun pengujian robot, terutama untuk simulasi.
Selain itu, dengan adanya ROS 2, controller robot yang diuji melalui simulasi bisa dengan mudah dipindahkan ke robot fisik untuk diuji secara langsung pada pengguna \citep{Takaya2016}.

\subsection{Rumusan Masalah}

Dari pemaparan yang telah dijelaskan di bagian latar belakang, dapat disimpulkan bahwa pengujian \emph{SARs} secara langsung pada manusia memiliki resiko yang besar terhadap keselamatan manusia serta besarnya biaya yang diperlukan untuk pengembangan dan pengujian robot.

\subsection{Penelitian Terkait}

Takaya et al. \citep{Takaya2016} mengembangkan lingkungan simulasi untuk pengujian terhadap \emph{mobile robot} menggunakan \emph{ROS} dan \emph{Gazebo}.
Sebelumnya penelitian yang sama juga dilakukan oleh Qian et al. \citep{Qian2014} untuk robot berjenis \emph{manipulator} dan Zhang et al. \citep{Zhang2015} untuk robot berjenis \emph{quadrotor UAV}.

Erickson et al. \citep{Erickson2020} mengembangkan \emph{framework} simulasi berbasis \emph{OpenAI Gym} \citep{Brockman2016} untuk \emph{assistive robotics}.
\emph{Framework} simulasi tersebut kemudian digunakan oleh Clegg et al. \citep{Clegg2020} untuk mengembangkan metode \emph{learning} pada kolaborasi antara robot dengan manusia dalam membantu pemakaian baju pada manusia.
Selain itu penelitian yang dilakukan Zamora et al. \citep{Zamora2016} menunjukkan simulasi pada \emph{OpenAI Gym} bisa digunakan bersamaan dengan \emph{ROS} dan \emph{Gazebo}.

\subsection{Tujuan Penelitian}

Tujuan dari penelitian ini adalah untuk mengembangkan lingkungan simulasi yang bisa digunakan untuk melakukan pengujian terhadap \emph{SARs} secara virtual, sebagai alternatif dari pengujian terhadap \emph{SARs} secara langsung.