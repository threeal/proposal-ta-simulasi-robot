\section{PENDAHULUAN}

\subsection{Latar Belakang}

Selama beberapa tahun belakang, robot telah mengalami perkembangan yang cukup signifikan dari robot besar untuk industri hingga robot kecil yang membantu pekerjaan ringan rumah tangga.
Salah satu jenis robot yang belakangan ini mulai banyak dikembangkan tersebut adalah \emph{socially assistive robots} (\emph{SARs}).
\emph{SARs} sendiri merupakan jenis robot yang membantu manusia melalui interaksi sosial seperti sebagai robot untuk pendamping maupun pelayan.

Dalam pengembangannya, seringkali pengujian suatu robot, termasuk \emph{SARs}, mengalami kendala karena pengujian secara langsung beresiko merusak \emph{hardware} yang mahal.
Selain itu robot yang dikembangkan juga akan lebih sulit untuk diubah desainnya karena pengubahan desain pada robot langsung akan memakan lebih banyak waktu dan biaya.

Salah satu solusi untuk mengatasi masalah tersebut adalah dengan menggunakan simulasi robot.
Simulasi robot sendiri merupakan simulasi yang digunakan untuk mensimulasikan \emph{physical robot} tanpa bergantung terhadap \emph{hardware} dari robot tersebut, sehingga bisa menghemat biaya dan waktu.
Untuk saat ini, sudah banyak aplikasi yang bisa digunakan untuk mensimulasikan robot tersebut seperti \emph{Gazebo}, \emph{V-Rep}, \emph{Webots}, \emph{Isaac Sim}, dan lain sebagainya.

Namun kendala dari simulasi yang umumnya digunakan saat ini adalah ketika mengembangkan program robot yang ada di simulasi, untuk membawanya ke robot aslinya maka diperlukan pemrograman ulang untuk robot tersebut.
Walaupun begitu, beberapa simulasi seperti \emph{Webots} dan \emph{Isaac Sim} mengatasi masalah tersebut dengan membuat \emph{compiler} yang mampu mengubah program yang ada di simulasi ke program yang ada di robot.
Namun kendala dari metode \emph{compiler} tersebut adalah keterbatasan robot yang bisa di-\emph{compile}, hanya robot komersial seperti \emph{Nao}, \emph{Turtlebot}, dan lain sebagainya yang bisa diubah dari simulasi ke robot aslinya.

Untuk mengatasi masalah tersebut, terutama untuk pengembangan robot kustom yang dibuat sendiri, maka di penelitian ini saya merumuskan desain dan implementasi simulasi terutama untuk \emph{socially assistive robots} sehingga program yang ada di robot tersebut dapat dengan mudah digunakan di simulasi maupun robot aslinya.

\subsection{Rumusan Masalah}

Dari pemaparan yang telah dijelaskan di bagian latar belakang, dapat disimpulkan bahwa pengujian \emph{SARs} secara langsung pada manusia memiliki resiko yang besar terhadap keselamatan manusia serta besarnya biaya yang diperlukan untuk pengembangan dan pengujian robot.

\subsection{Penelitian Terkait}

Takaya et al. \citep{Takaya2016} mengembangkan lingkungan simulasi untuk pengujian terhadap \emph{mobile robot} menggunakan \emph{ROS} dan \emph{Gazebo}.
Sebelumnya penelitian yang sama juga dilakukan oleh Qian et al. \citep{Qian2014} untuk robot berjenis \emph{manipulator} dan Zhang et al. \citep{Zhang2015} untuk robot berjenis \emph{quadrotor UAV}.

Erickson et al. \citep{Erickson2020} mengembangkan \emph{framework} simulasi berbasis \emph{OpenAI Gym} \citep{Brockman2016} untuk \emph{assistive robotics}.
\emph{Framework} simulasi tersebut kemudian digunakan oleh Clegg et al. \citep{Clegg2020} untuk mengembangkan metode \emph{learning} pada kolaborasi antara robot dengan manusia dalam membantu pemakaian baju pada manusia.
Selain itu penelitian yang dilakukan Zamora et al. \citep{Zamora2016} menunjukkan simulasi pada \emph{OpenAI Gym} bisa digunakan bersamaan dengan \emph{ROS} dan \emph{Gazebo}.

\subsection{Tujuan Penelitian}

Tujuan dari penelitian ini adalah untuk mengembangkan lingkungan simulasi yang bisa digunakan untuk melakukan pengujian terhadap \emph{SARs} secara virtual, sebagai alternatif dari pengujian terhadap \emph{SARs} secara langsung.