\section{HASIL YANG DIHARAPKAN}

Dari penelitian yang telah dilakukan, diharapkan akan menghasilkan sebuah lingkungan simulasi yang mampu digunakan untuk melakukan pengujian pada SARs secara virtual menggunakan simulasi robot.
Sehingga pengujian yang dilakukan dapat meminimalisir resiko, mengurangi biaya, serta menghemat waktu dalam pengembangan SARs.

\section{RENCANA KERJA}

\newcommand{\w}{}
\newcommand{\G}{\cellcolor{gray}}
\begin{table}[h!]
  \begin{tabular}{|p{3.5cm}|c|c|c|c|c|c|c|c|c|c|c|c|c|c|c|c|}

    \hline
    \multirow{2}{*}{Kegiatan} & \multicolumn{16}{|c|}{Minggu} \\
    \cline{2-17} &
    1 & 2 & 3 & 4 & 5 & 6 & 7 & 8 & 9 & 10 & 11 & 12 & 13 & 14 & 15 & 16 \\
    \hline

    Pengembangan controller robot &
    \G & \G & \G & \G & \G & \G & \w & \w & \w & \w & \w & \w & \w & \w & \w & \w \\
    \hline

    Pembuatan lingkungan simulasi &
    \w & \w & \w & \w & \w & \w & \G & \G & \G & \G & \w & \w & \w & \w & \w & \w \\
    \hline

    Pengujian robot pada lingkungan simulasi &
    \w & \w & \w & \w & \w & \w & \w & \w & \w & \w & \G & \G & \G & \G & \w & \w \\
    \hline

    Evaluasi hasil pengujian &
    \w & \w & \w & \w & \w & \w & \w & \w & \w & \w & \w & \w & \w & \w & \G & \G \\
    \hline

  \end{tabular}
\end{table}
