\section{HASIL YANG DIHARAPKAN}

Dari penelitian yang telah dilakukan, diharapkan akan menghasilkan sebuah lingkungan simulasi yang mampu digunakan untuk menguji keseluruhan fungsionalitas yang ada pada \emph{SARs} sehingga pengujian terhadap \emph{SARs} dapat dilakukan secara virtual.

\section{RENCANA KERJA}

\newcommand{\w}{}
\newcommand{\G}{\cellcolor{gray}}
\begin{table}[h!]
  \begin{tabular}{|p{3.5cm}|c|c|c|c|c|c|c|c|c|c|c|c|c|c|c|c|}

    \hline
    \multirow{2}{*}{Kegiatan} & \multicolumn{16}{|c|}{Minggu} \\
    \cline{2-17} &
    1 & 2 & 3 & 4 & 5 & 6 & 7 & 8 & 9 & 10 & 11 & 12 & 13 & 14 & 15 & 16 \\
    \hline

    Pengembangan sistem Robot &
    \G & \G & \G & \G & \G & \G & \w & \w & \w & \w & \w & \w & \w & \w & \w & \w \\
    \hline

    Pembuatan lingkungan simulasi &
    \w & \w & \w & \w & \w & \w & \G & \G & \G & \G & \w & \w & \w & \w & \w & \w \\
    \hline

    Pengujian sistem robot terhadap lingkungan simulasi &
    \w & \w & \w & \w & \w & \w & \w & \w & \w & \w & \G & \G & \G & \G & \w & \w \\
    \hline

    Evaluasi hasil pengujian &
    \w & \w & \w & \w & \w & \w & \w & \w & \w & \w & \w & \w & \w & \w & \G & \G \\
    \hline

  \end{tabular}
\end{table}