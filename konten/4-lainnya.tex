\section{HASIL YANG DIHARAPKAN}

\subsection{Hasil yang Diharapkan dari Penelitian}

Dari penelitian yang akan dilakukan, diharapkan dapat menghasilkan sebuah lingkungan simulasi yang mampu digunakan untuk melakukan pengujian pada \emph{socially assistive robots} (SARs) secara virtual menggunakan simulasi robot, sebagai alternatif dari pengujian yang secara langsung menguji pengguna dengan robot fisik.
Dari penelitian ini juga diharapkan pengembangan pada SARs bisa dilakukan dengan lebih mudah karena dengan adanya lingkungan simulasi, pengujian yang dilakukan pada SARs dapat dilakukan dengan resiko yang lebih minim, biaya yang lebih murah, serta waktu yang lebih cepat.

\subsection{Hasil Pendahuluan}

Desain 3D berbasis CAD dari robot yang akan kami gunakan sudah ada seperti yang terlihat pada \ref{fig:RobotDesign}, yang perlu dilakukan adalah mengubah desain tersebut mengikuti format SDF yang ada pada simulator Gazebo.
Untuk simulator Gazebo tersebut sendiri sudah kami coba menggunakan contoh lingkungan simulasi yang sudah ada.
Dari lingkungan simulasi tersebut nantinya akan dikembangkan lebih lanjut menjadi lingkungan simulasi yang menyesuaikan dengan pengujian yang akan dilakukan.

Terkait dengan kontroler robot untuk simulasi, sebelumnya kontroler serupa sudah pernah kami buat menggunakan ROS 2 untuk simulator Webots dengan desain robot yang lain.
Struktur yang ada pada kontroler tersebut juga sama dimana terpisah menjadi sebuah node behavior dan beberapa node lain yang memproses sensor maupun aktuator yang ada di lingkungan simulasi.
Yang perlu dilakukan adalah menyesuaikan kontroler dengan struktur tersebut agar bisa digunakan pada simulator Gazebo, serta mengatur sensor dan aktuator yang digunakan di simulasi menyesuaikan sensor dan aktuator yang digunakan pada robot yang akan diujikan di penelitian ini.

\section{RENCANA KERJA}

\newcommand{\w}{}
\newcommand{\G}{\cellcolor{gray}}
\begin{table}[h!]
  \begin{tabular}{|p{42mm}|c|c|c|c|c|c|c|c|c|c|c|c|c|c|c|c|}

    \hline
    \multirow{2}{*}{Kegiatan} & \multicolumn{16}{|c|}{Minggu} \\
    \cline{2-17} &
    1 & 2 & 3 & 4 & 5 & 6 & 7 & 8 & 9 & 10 & 11 & 12 & 13 & 14 & 15 & 16 \\
    \hline

    Pembuatan lingkungan simulasi &
    \G & \G & \G & \G & \G & \G & \w & \w & \w & \w & \w & \w & \w & \w & \w & \w \\
    \hline

    Pengembangan kontroler robot &
    \w & \w & \G & \G & \G & \G & \G & \G & \G & \G & \G & \G & \G & \G & \w & \w \\
    \hline

    Pengujian robot pada lingkungan simulasi &
    \w & \w & \w & \w & \w & \w & \w & \w & \G & \G & \G & \G & \w & \w & \w & \w \\
    \hline

    Pemindahan kontroler ke robot fisik &
    \w & \w & \w & \w & \w & \w & \w & \w & \w & \w & \w & \w & \G & \G & \w & \w \\
    \hline

    Evaluasi hasil pengujian &
    \w & \w & \w & \w & \w & \w & \w & \w & \w & \w & \w & \w & \w & \w & \G & \G \\
    \hline

  \end{tabular}
\end{table}
