\begin{flushleft}
  \textbf{Departemen Teknik Komputer - FTEIC}\\
  \textbf{Institut Teknologi Sepuluh Nopember}\\
\end{flushleft}

\begin{center}
  \underline{\textbf{EC184701 - PRA TUGAS AKHIR - 2 SKS}}
\end{center}

\begin{adjustwidth}{-0.2cm}{}
  \begin{tabular}{lcp{0.7\linewidth}}

    Nama Mahasiswa &:& Muhammad Alfi Maulana Fikri \\
    Nomor Pokok &:&	0721 17 4000 0009 \\

    Semester &:& Ganjil 2020/2021 \\

    Dosen Pembimbing &:& 1. Prof. Dr. Ir. Mauridhi Hery Purnomo, M.Eng. \\
    & & 2. Dr. I Ketut Eddy Purnama, S.T., M.T. \\

    Judul Tugas Akhir &:& \textbf{Pengembangan Lingkungan Simulasi untuk Pengujian} \\
    & & \textbf{\emph{Socially Assistive Robots} Menggunakan ROS 2 dan Gazebo} \\

    Uraian Tugas Akhir &:& \\
  \end{tabular}
\end{adjustwidth}

Pada penelitian ini kami mengajukan lingkungan simulasi untuk pengujian \emph{Socially Assistive Robots} (SARs) yang dikembangkan menggunakan ROS 2 dan Gazebo.
Di dalam lingkungan simulasi ini, model robot yang digunakan akan diujikan dengan model manusia virtual serta model-model objek lain yang ada.
Untuk mempermudah perpindahan program dari simulasi ke robot asli, controller robot akan dikembangkan secara terpisah dari lingkungan simulasi yang ketika pengujian, keduanya akan saling terhubung menggunakan sistem komunikasi antar proses yang ada di ROS 2.
Diharapkan lingkungan simulasi yang dibuat dapat membantu pengujian SARs dengan meminimalisir resiko, mengurangi biaya, dan menghemat waktu jika dibandingkan dengan melakukan pengujian secara langsung menggunakan robot asli.
\vspace{1ex}

\begin{flushright}
  Surabaya, Januari 2021
\end{flushright}
\vspace{1ex}

\begin{center}

  \begin{multicols}{2}

    Dosen Pembimbing 1
    \vspace{12ex}

    \underline{Prof. Dr. Ir. Mauridhi Hery Purnomo, M.Eng.} \\
    NIP. 19580916 198601 1 001

    \columnbreak

    Dosen Pembimbing 2
    \vspace{12ex}

    \underline{Dr. I Ketut Eddy Purnama, S.T., M.T.} \\
    NIP. 19690730 199512 1 001

  \end{multicols}
  \vspace{6ex}

  Mengetahui, \\
  Kepala Departemen Teknik Komputer FTEIC - ITS
  \vspace{12ex}

  \underline{Dr. Supeno Mardi Susiki Nugroho, S.T., M.T.} \\
  NIP. 19700313 199512 1 001

\end{center}
