\begin{flushleft}
  \textbf{Departemen Teknik Komputer - FTEIC}\\
  \textbf{Institut Teknologi Sepuluh Nopember}\\
\end{flushleft}

\begin{center}
  \underline{\textbf{EC184701 - PRA TUGAS AKHIR - 2 SKS}}
\end{center}

\begin{adjustwidth}{-0.2cm}{}
  \begin{tabular}{lcp{0.7\linewidth}}

    Nama Mahasiswa &:& Muhammad Alfi Maulana Fikri \\
    Nomor Pokok &:&	0721 17 4000 0009 \\

    Semester &:& Ganjil 2020/2021 \\

    Calon Dosen Pembimbing &:& 1. Prof. Dr. Ir. Mauridhi Hery Purnomo, M.Eng. \\
    & & 2. Dr. I Ketut Eddy Purnama, S.T., M.T. \\

    Judul Tugas Akhir &:& \textbf{Pengembangan Lingkungan Simulasi untuk Pengujian} \\
    & & \textbf{\emph{Socially Assistive Robots} (\emph{SARs}) Menggunakan \emph{ROS 2}} \\
    & & \textbf{dan \emph{Gazebo}} \\

    Uraian Tugas Akhir &:& \\
  \end{tabular}
\end{adjustwidth}

Selama beberapa tahun belakang, robot telah mengalami perkembangan yang cukup signifikan dari robot besar untuk industri hingga robot kecil yang membantu pekerjaan ringan rumah tangga.
Salah satu jenis robot yang belakangan ini mulai banyak dikembangkan tersebut adalah \emph{socially assistive robots} (\emph{SARs}).
Dalam pengembangannya, seringkali pengujian suatu robot, termasuk \emph{SARs}, mengalami kendala karena pengujian secara langsung beresiko merusak \emph{hardware} yang mahal.
Salah satu solusi untuk mengatasi masalah tersebut adalah dengan menggunakan simulasi robot.
Namun kendala dari simulasi yang umumnya digunakan saat ini adalah ketika mengembangkan program robot yang ada di simulasi, untuk membawanya ke robot aslinya maka diperlukan pemrograman ulang untuk robot tersebut.
Untuk mengatasi masalah tersebut, terutama untuk pengembangan robot kustom yang dibuat sendiri, maka di penelitian ini saya merumuskan desain dan implementasi simulasi terutama untuk \emph{socially assistive robots} sehingga program yang ada di robot tersebut dapat dengan mudah digunakan di simulasi maupun robot aslinya.
\vspace{1ex}

\begin{flushright}
  Surabaya, Desember 2020
\end{flushright}
\vspace{1ex}

\begin{center}

  \begin{multicols}{2}

    Calon Dosen Pembimbing 1
    \vspace{12ex}

    \underline{Prof. Dr. Ir. Mauridhi Hery Purnomo, M.Eng.} \\
    NIP. 19580916 198601 1 001

    \columnbreak

    Calon Dosen Pembimbing 2
    \vspace{12ex}

    \underline{Dr. I Ketut Eddy Purnama, S.T., M.T.} \\
    NIP. 19690730 199512 1 001

  \end{multicols}
  \vspace{6ex}

  Mengetahui, \\
  Kepala Departemen Teknik Komputer FTEIC - ITS
  \vspace{12ex}

  \underline{Dr. Supeno Mardi Susiki Nugroho, S.T., M.T.} \\
  NIP. 19700313 199512 1 001

\end{center}
