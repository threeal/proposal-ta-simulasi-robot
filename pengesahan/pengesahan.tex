\begin{flushleft}
  \textbf{Departemen Teknik Komputer - FTEIC}\\
  \textbf{Institut Teknologi Sepuluh Nopember}\\
\end{flushleft}

\begin{center}
  \underline{\textbf{EC184701 - PRA TUGAS AKHIR - 2 SKS}}
\end{center}

\begin{adjustwidth}{-0.2cm}{}
  \begin{tabular}{lcp{0.7\linewidth}}

    Nama Mahasiswa &:& Muhammad Alfi Maulana Fikri \\
    Nomor Pokok &:&	0721 17 4000 0009 \\

    Semester &:& Ganjil 2020/2021 \\

    Calon Dosen Pembimbing &:& 1. Prof. Dr. Ir. Mauridhi Hery Purnomo, M.Eng. \\
    & & 2. Dr. I Ketut Eddy Purnama, S.T., M.T. \\

    Judul Tugas Akhir &:& \textbf{Pengembangan Lingkungan Simulasi untuk Pengujian} \\
    & & \textbf{\emph{Socially Assistive Robots} Menggunakan \emph{ROS 2}} \\
    & & \textbf{dan \emph{Gazebo}} \\

    Uraian Tugas Akhir &:& \\
  \end{tabular}
\end{adjustwidth}

Socially assistive robots (SARs) merupakan jenis robot dalam bidang socially assistive robotics yang mampu memberikan bantuan kepada pengguna dalam bentuk interaksi sosial.
Namun, karena sifat dari SARs yang melibatkan interaksi langsung dengan pengguna, maka, pengujian dari robot akan menjadi sulit dan beresiko bagi pengguna yang ikut terlibat dalam pengujian tersebut.
Salah satu solusi untuk mengatasi masalah tersebut adalah dengan melakukan pengujian secara virtual melalui simulasi robot.
Selain bisa meminimalisir resiko, penggunaan simulasi robot juga bisa mengurangi biaya yang dibutuhkan dan menghemat waktu pengujian.
Hingga saat ini sudah ada beberapa simulator yang bisa digunakan untuk menjalankan simulasi robot.
Namun, simulator tersebut hanyalah platform yang secara umum digunakan untuk membantu pengembangan robot melalui simulasi virtual, sedangkan pengembangan dari lingkungan simulasi dan controller robot untuk simulasi tersebut harus dibuat sendiri oleh pengembang robot.
Untuk itu, pada tugas akhir ini saya mengajukan penelitian terkait pengembangan lingkungan simulasi untuk pengujian SARs menggunakan ROS 2 dan Gazebo.
Dari penelitian ini diharapkan adanya lingkungan simulasi dan controller robot yang bisa digunakan untuk melakukan pengujian SARs secara virtual menggunakan simulasi robot.
\vspace{1ex}

\begin{flushright}
  Surabaya, Desember 2020
\end{flushright}
\vspace{1ex}

\begin{center}

  \begin{multicols}{2}

    Calon Dosen Pembimbing 1
    \vspace{12ex}

    \underline{Prof. Dr. Ir. Mauridhi Hery Purnomo, M.Eng.} \\
    NIP. 19580916 198601 1 001

    \columnbreak

    Calon Dosen Pembimbing 2
    \vspace{12ex}

    \underline{Dr. I Ketut Eddy Purnama, S.T., M.T.} \\
    NIP. 19690730 199512 1 001

  \end{multicols}
  \vspace{6ex}

  Mengetahui, \\
  Kepala Departemen Teknik Komputer FTEIC - ITS
  \vspace{12ex}

  \underline{Dr. Supeno Mardi Susiki Nugroho, S.T., M.T.} \\
  NIP. 19700313 199512 1 001

\end{center}
