\begin{flushleft}
  % Ubah kalimat berikut sesuai dengan nama departemen dan fakultas
  \textbf{Departemen Teknik Komputer - FTEIC}\\
  % Ubah kalimat berikut sesuai dengan nama universitas
  \textbf{Institut Teknologi Sepuluh Nopember}\\
\end{flushleft}

\begin{center}
  % Ubah detail mata kuliah berikut sesuai dengan yang ditentukan oleh departemen
  \underline{\textbf{EC184701 - PRA TUGAS AKHIR - 2 SKS}}
\end{center}

\begin{adjustwidth}{-0.2cm}{}
  \begin{tabular}{lcp{0.7\linewidth}}

    % Ubah kalimat-kalimat berikut sesuai dengan nama dan NRP mahasiswa
    Nama Mahasiswa &:& Muhammad Alfi Maulana Fikri \\
    Nomor Pokok &:&	0721 17 4000 0009 \\

    % Ubah kalimat berikut sesuai dengan semester pengajuan proposal
    Semester &:& Ganjil 2020/2021 \\

    % Ubah kalimat-kalimat berikut sesuai dengan nama-nama dosen pembimbing
    Dosen Pembimbing &:& 1. Prof. Dr. Ir. Mauridhi Hery Purnomo, M.Eng. \\
    & & 2. Dr. I Ketut Eddy Purnama, ST., MT. \\

    % Ubah kalimat berikut sesuai dengan judul tugas akhir
    Judul Tugas Akhir &:& \textbf{Pengembangan Simulasi untuk \emph{Socially Assistive Robots} dengan Mengintegrasikan \emph{ROS 2} di \emph{Gazebo}} \\

    Uraian Tugas Akhir &:& \\
  \end{tabular}
\end{adjustwidth}

% Ubah paragraf berikut sesuai dengan uraian dari tugas akhir
Selama beberapa tahun belakang, robot telah mengalami perkembangan yang cukup signifikan dari robot besar untuk industri hingga robot kecil yang membantu pekerjaan ringan rumah tangga.
Salah satu jenis robot yang belakangan ini mulai banyak dikembangkan tersebut adalah \emph{socially assistive robots} (\emph{SARs}).
Dalam pengembangannya, seringkali pengujian suatu robot, termasuk \emph{SARs}, mengalami kendala karena pengujian secara langsung beresiko merusak \emph{hardware} yang mahal.
Salah satu solusi untuk mengatasi masalah tersebut adalah dengan menggunakan simulasi robot.
Namun kendala dari simulasi yang umumnya digunakan saat ini adalah ketika mengembangkan program robot yang ada di simulasi, untuk membawanya ke robot aslinya maka diperlukan pemrograman ulang untuk robot tersebut.
Untuk mengatasi masalah tersebut, terutama untuk pengembangan robot kustom yang dibuat sendiri, maka di penelitian ini saya merumuskan desain dan implementasi simulasi terutama untuk \emph{socially assistive robots} sehingga program yang ada di robot tersebut dapat dengan mudah digunakan di simulasi maupun robot aslinya.
\vspace{1ex}

\begin{flushright}
  % Ubah kalimat berikut sesuai dengan tempat, bulan, dan tahun penulisan
  Surabaya, Desember 2020
\end{flushright}
\vspace{1ex}

\begin{center}

  \begin{multicols}{2}

    Dosen Pembimbing 1
    \vspace{12ex}

    % Ubah kalimat-kalimat berikut sesuai dengan nama dan NIP dosen pembimbing pertama
    \underline{Prof. Dr. Ir. Mauridhi Hery P., M.Eng.} \\
    NIP. 19580916 198601 1 001

    \columnbreak

    Dosen Pembimbing 2
    \vspace{12ex}

    % Ubah kalimat-kalimat berikut sesuai dengan nama dan NIP dosen pembimbing kedua
    \underline{Dr. I Ketut Eddy P., S.T., M.T.} \\
    NIP. 19690730 199512 1 001

  \end{multicols}
  \vspace{6ex}

  Mengetahui, \\
  % Ubah kalimat berikut sesuai dengan jabatan kepala departemen
  Kepala Departemen Teknik Komputer FTEIC - ITS
  \vspace{12ex}

  % Ubah kalimat-kalimat berikut sesuai dengan nama dan NIP kepala departemen
  \underline{Dr. Supeno Mardi Susiki Nugroho, S.T., M.T.} \\
  NIP. 19700313 199512 1 001

\end{center}
